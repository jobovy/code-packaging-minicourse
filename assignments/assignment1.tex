\documentclass[12pt]{article}
\usepackage[letterpaper,margin=1in]{geometry}
\usepackage{hyperref}
\usepackage{amsmath}
\begin{document}
\begin{center}
{\bf \LARGE AST3100 ``Code development and packaging mini-course'' Assignment 1}\\[7pt]
\emph{Due on Mar. 3 2020  at the start of class}\\[7pt]
\end{center}

The purpose of this first assignment is to start developing the Python
package that you will create as part of this course and to play around
with \texttt{git} and GitHub. To submit the assignment, please send a
link to your package's GitHub page to the instructor and he will
monitor your progress there (he will be looking for lots of commits
\ldots).\\

\noindent{\bf Task 1: Create your package} Following the notes, create
the absolute basic skeleton of your package: (a) choose a name; (b)
the start of your package's code (but only a very basic start, we will
add the rest later; so something like a single file and a single,
small function); (c) a basic \texttt{setup.py} file. Test that you can
install your code by running\\

\texttt{python setup.py develop}\\

\noindent Don't add \texttt{README} or \texttt{LICENSE} files at this point, we
will do that in the next step.\\

\noindent{\bf Task 2: Start your package's GitHub page} Go to your
GitHub account and create a new repository for your package, making
sure to initialize it with a \texttt{README.md}, a \texttt{.gitignore}
file, and a license.\\

\noindent{\bf Task 3: Start the \texttt{git} commit history for your
  package} Clone your package's GitHub repository to your local
machine (use a different directory from where you started the
package). Copy the skeleton that you created in Task 1 to this
directory and add and commit the skeleton that you created earlier,
keeping to the \texttt{git} philosophy of committing early and
often. When you have done this, push the changes to GitHub; check that
they appear correctly there.\\

\noindent{\bf Task 4: Implement a first feature in your package} Now
you should implement a first full-fledged feature in your package,
some of its basic functionality. Start a new branch with a good name
for the feature and start implementing it, committing liberally along
the way. Make sure to add a bug somewhere along the way for the next
task. Push your changes to GitHub from time to time. When the feature
is finished, merge it into the \texttt{master} branch from the
command-line.\\

\noindent{\bf Task 5: Open an issue on GitHub} The feature that you
implemented in the previous task has a bug. Pretend that you just
found it and open an Issue on the GitHub page describing the issue as
discussed in the notes.\\

\noindent{\bf Task 6: Fix the issue through a pull request} Next, fix
the bug reported in the previous task. Start a new branch to fix the
bug and fix it (referring to the issue by its number as \texttt{\#1} in
each commit). Once the bug has been fixed, open a pull request to
merge the fix back into the \texttt{master} branch.

\end{document}
