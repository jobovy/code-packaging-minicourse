\documentclass[12pt]{article}
\usepackage[letterpaper,margin=1in]{geometry}
\usepackage{hyperref}
\usepackage{amsmath}
\begin{document}
\begin{center}
{\bf \LARGE AST3100 ``Code development and packaging mini-course'' Assignment 3}\\[7pt]
\emph{Due on Mar. 17 2020  at the start of class}\\[7pt]
\end{center}

The purpose of this third assignment is to build a test suite for the
Python package that you are creating as part of this course, run it
with \texttt{pytest}, and set it up to run automatically on
\texttt{Travis CI}. Write your tests in a \texttt{tests/}
sub-directory of your top-level directory as discussed in the notes.\\

\noindent{\bf Task 1: Write tests for your code that check that in
  cases where you know the answer, your code returns the correct
  answer} Do this for some of the functions and methods in your
package. If you think that none of the functions in your package have
a known solution, then you need to think hard (because there probably
is one) or re-design your package such that it consists of functions
for which some known solutions exist.\\

\noindent{\bf Task 2: Write tests for your code that check that the
  outputs that your functions and methods provide are consistent with
  known properties of the outputs} Even if you don't know the answer,
what property do you expect the output to have? This can be as simple
as checking the expected range of the output, or that the output
satisfies some constraints.\\

\noindent{\bf Task 3: Install \texttt{coverage.py} and run your tests
  with \texttt{coverage} and \texttt{pytest} to determine the test
  coverage of your test suite} following the steps given in the notes.\\

\noindent{\bf Task 4: Setup your project on \texttt{Travis CI} and run
  the build-and-test integration automatically} Write a
\texttt{.travis.yml} file that installs all dependencies (with
\texttt{pip} if you can), installs your code, and runs the test suite,
displaying the test coverage.\\

\noindent{\bf Task 5: Use Miniconda on \texttt{Travis CI} to install
  dependencies} and run again.\\

\noindent{\bf Task 6 (extra credit): Setup your project on
  \texttt{AppVeyor} and run the build-and-test integration
  automatically} by writing a \texttt{.appveyor.yml} file and adding
it to your repository.\\

\end{document}
