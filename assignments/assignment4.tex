\documentclass[12pt]{article}
\usepackage[letterpaper,margin=1in]{geometry}
\usepackage{hyperref}
\usepackage{amsmath}
\setlength{\emergencystretch}{2em}
\begin{document}
\begin{center}
{\bf \LARGE AST3100 ``Code development and packaging mini-course'' Assignment 4}\\[7pt]
\emph{Due on Mar. 24 2020  at the start of class}\\[7pt]
\end{center}

The purpose of this fourth (and final!) assignment is to set up your
documentation on \texttt{readthedocs.io} and to release your package
to \texttt{PyPI}. Note that if you don't feel ready to release the
first version of your package yet, you can skip the final part, but
please do upload a first version (could be ``0.0.1'') to
\texttt{TestPyPI}.\\

\noindent{\bf Task 1: Set up your documentation on
  \texttt{readthedocs.io}} following the instructions in the notes.\\

\noindent{\bf Task 2: Prepare for the first release of your package}
by deciding on a proper version scheme and making sure your version
strings in \texttt{setup.py}, \texttt{\_\_init\_\_.py}, and
\texttt{docs/source/conf.py} are consistent. Also add a
\texttt{HISTORY.md} file with the history of your package and a
\texttt{MANIFEST.in} file that makes sure to include the
\texttt{LICENSE} file in your source distribution.\\

\noindent{\bf Task 3: Create the source distribution} and unpack it somewhere else to check that all looks okay.\\

\noindent{\bf Task 4: Register for an account on \texttt{TestPyPI} and
  release your first version there.} Upload the source
distribution. Please send your instructor a link to the
\texttt{TestPyPI} page.\\

\noindent{\bf Task 5: Build a wheel for your package and add that to
  your \texttt{TestPyPI} release.}.\\

\noindent{\bf Task 6 (optional): Release your source distribution and
  wheel to \texttt{PyPI}.}

\end{document}
